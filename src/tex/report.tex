\documentclass[letterpaper,sigplan]{acmart}
\usepackage{listings}
\usepackage[utf8]{inputenc}
\usepackage[T1]{fontenc}
\usepackage{graphicx}
\usepackage{grffile}
\usepackage{longtable}
\usepackage{wrapfig}
\usepackage{rotating}
\usepackage[normalem]{ulem}
\usepackage{amsmath}
\usepackage{textcomp}
\usepackage{amssymb}
\usepackage{capt-of}
\usepackage{microtype}
\usepackage[htt]{hyphenat}
\usepackage{multirow}
\usepackage{booktabs}
\usepackage{hyperref}
\title{Testing dk.brics.automaton and dk.brics.string}
\date{\today}
\hypersetup{
  pdfauthor={Marlin Roberts \& Kenny Ballou},
  pdftitle={Project Report},
  pdfkeywords={},
  pdfsubject={},
  pdflang={English}}

\AtBeginDocument{%
  \providecommand\BibTeX{{%
      \normalfont B\kern-0.5em{\scshape i\kern-0.25em b}\kern-0.8em\TeX}}}

\setcopyright{iw3c2w3g}
\copyrightyear{2020}
%\acmYear{2018}
\acmDOI{10.1145/0000000000}

\author{Marlin Roberts}
\email{marlinroberts@u.boisestate.edu}
\affiliation{%
  \institution{Boise State University}
}

\author{Kenny Ballou}
\email{kennyballou@u.boisestate.edu}
\affiliation{%
  \institution{Boise State University}
}

\begin{CCSXML}
  <ccs2012>
  <concept>
  <concept_id>10011007.10011006.10011073</concept_id>
  <concept_desc>Software and its engineering~Software maintenance tools</concept_desc>
  <concept_significance>500</concept_significance>
  </concept>
  <concept>
  <concept_id>10011007.10011074.10011099</concept_id>
  <concept_desc>Software and its engineering~Software verification and validation</concept_desc>
  <concept_significance>500</concept_significance>
  </concept>
  </ccs2012>
\end{CCSXML}

\ccsdesc[500]{Software and its engineering~Software maintenance tools}
\ccsdesc[500]{Software and its engineering~Software verification and validation}

%% https://tex.stackexchange.com/questions/205362/should-i-use-lstinline-for-the-language-keywords-embedded-in-text
\lstset{
  frameround=fttt,
  language=Java,
  numbers=left,
  breaklines=true,
  keywordstyle=\color{blue}\bfseries,
  basicstyle=\ttfamily,
  numberstyle=\color{black}
}

\begin{document}

\begin{abstract}
  The \lstinline{string}~\cite{strings2003} and
  \lstinline{automaton} libraries were examined and tested to study and perform
  property verification.  Various assertions and code instrumentation were
  added to the two libraries to verify several state and sequential properties.
  Several possible bugs and recommendations will be discussed as a result of
  this examination.
\end{abstract}

\maketitle{}
% \tableofcontents{}

\section{Introduction}

The Java String Analyzer~\cite{strings2003} tool consists of two main
libraries, \lstinline{string} and \lstinline{automaton}.  The
\lstinline{automaton} library implements a deterministic and non-deterministic
finite-state automata with an unrestricted set of regular expression operators.
The \lstinline{string} library uses the \lstinline{automaton} library to model
various Java \lstinline{String}~\cite{oracle-java-string} and
\lstinline{StringBuilder}~\cite{oracle-java-stringbuilder} operations.

The two libraries were examined, instrumented, and tested to verify various
program properties.  Several tools were used to perform the verification,
chiefly, Java \lstinline{Assertions}~\cite{oracle-java-assertions},
\lstinline{JUnit}~\cite{junit4}, and \lstinline{JavaMOP}~\cite{javamop}.

The following is a brief overview of the paper.  First the methodology of the
approach to property discover, property verification, and program
testing (\ref{sec:methodology}).  Next, the program properties
examined (\ref{sec:properties}).  Then, the various aspects of testing the two
libraries (\ref{sec:testing}).  Before concluding, the analysis and results of
our program verification study (\ref{sec:results}).

\section{Methodology}\label{sec:methodology}

\subsection{Property Discovery}

Three approaches were used to identify interesting properties: Reading
documentation within the source code or in \texttt{README} files, examining the
source code directly, and observing event sequences.  Each approach identified
interesting properties.

The \lstinline{automaton} package contained comments that directly referenced
the state of a returned automata, but few were found in the string operations
with similar detail.  Automata operations that returned determinized or
minimized automata were usually annotated with comments indicating so.  Some
properties were determined by the documented purpose of the string operations,
which is to model the behavior of Java \lstinline{String} and
\lstinline{StringBuilder} operations.  The expected behaviors of Java
\lstinline{String} and \lstinline{StringBuilder} operations are available from
online sources~\cite{oracle-java-string,oracle-java-stringbuilder}.  These
behaviors were examined to determine additional properties that related
directly to the correct operation of the software under test.

Examination of the source code provided several interesting properties.  An
example is the discovery of an assertion in one of the string operations, which
asserted that the input automata was deterministic.  This shows that the
programmers expected a certain set of conditions to exist even if they did not
explicitly indicate so in the documentation.  Interestingly, test cases that
should have provided a deterministic automata failed this assertion and led to
the conclusion that some operations for building automata might produce
automata with incorrect or inconsistent properties.

Observing event sequences during preliminary testing showed interesting
sequential properties.  For instance, a majority of the string operations ended
with calls to \lstinline{Automaton.clearHashCode()} followed by
\lstinline{Automaton.recomputeHashCode()}, while one did not.  This indicated
that static checks should be put in place to determine if the hash code of
returned automata was correct, however, this proved problematic due to the
visibility of the instance variable representing the hash code and the behavior
of the method that returns the hash code. Sequence checking was then put in
place in order to observe this property to the extent possible.  Section
~\ref{sec:properties}, Properties, contains the complete details of identified
properties.

\subsection{Property Checking}

The primary methods for checking state or static properties are Java Modeling
Language~\cite{openjml} (JML) and Java
assertions~\cite{oracle-java-assertions}.  JML has significant advantages over
assertions, including the ability to easily compare \textit{pre} and
\textit{post} conditions of variables and results.  Using JML requires the use
of a specific Java compiler that translates the JML annotations into regular
Java bytecode for execution.  Attempting to use this compiler exposed
significant issues that prevented successful compilation.  While workarounds
did produce a compiled project, further errors were encountered during
execution that made continued use of JML too problematic.

Java assertions were then implemented in order to check the state properties.
The assertions are located in test cases where possible, in order to minimize
intrusion into the classes being observed.  However, in cases where properties
were not visible to the test cases, assertions were placed in the source code.
Results of static property checking are discussed in Section~\ref{sec:results}.

Checking for sequential properties was accomplished with the
Monitoring-Oriented Programming (MOP) tool, Java MOP~\cite{javamop}, along with
the AspectJ~\cite{eclipse-aspectj} compiler.  Checks provided by these tools
``cut-across'' classes within the software under test which allows observation
of events generated anywhere in the source code. These tools were initially
used to identify interesting sequential properties by defining a wide range of
possible events and observing the sequences produced during preliminary
testing.  After these properties were identified, the MOP definition files were
modified to observe only the events related to these properties.  Logic was
introduced to check for correct or incorrect sequences.  Results of sequential
property checking are discussed in Section~\ref{sec:results}.

\subsection{Static Analysis}

Static analysis techniques were applied using
Coverity~\cite{coverity-scan}.  This online static analysis tool provides the
ability to run a variety of checks that are provided in sets called
checkers.  Checkers are specific to a source language and are further targeted
at specific program types.  For instance, a Java program that provides a web
service may have a checker that examines the source for specific issues and
code smells that commonly appear in web services code.  The source code is
prepared for analysis by performing a build using supplied tools that records
the build operation and results.  This information is then submitted for
analysis.

An abundance of false positives tend to be a significant issue when using
static analysis tools.  False positives are code problems identified by static
analysis tools, but in fact are not.  Tools like Coverity allow for analysis
tuning, which is the process of changing the scope of the analysis or the
definition of what constitutes an actual problem.

Results of the static analysis are discussed in Section~\ref{sec:results}.


\section{Properties}\label{sec:properties}

The majority of the properties identified for the string operations involve the
ability of the operation to match the behavior of the Java String or
StringBuilder operation it models.  Due to the large number of operations and
the related behaviors, this set of properties was treated as an opportunity to
apply some of the regression testing techniques, with the idea that the
operations likely conform to those behaviors now and tests should be available
to make sure they continue to operate correctly.  Details of this process is
provided in Section~\ref{sec:testing}.

Properties identified for the \lstinline{automaton} package relate to the
structure and validity of the automata they return.  Some of these properties
could be checked by using available accessor methods, such as whether or not
automata are deterministic being available through the
\lstinline{Automaton.isDeterministic()} method.  Other properties, such as
whether or not an automata is minimal, did not have a readily accessible flag
for testing.  Various approaches were discussed as to how a property like this
could be tested.  One approach would be to make a clone of a given automata
that should be minimal, minimizing it and comparing it with the original.  If
the two were equivalent, then the original automata was likely minimal to begin
with.  Another approach was to monitor sequences of events in order to
determine if the automata had the minimize operation performed on it before
being returned.  In this particular case it was decided that monitoring the
sequence of method calls was a less intrusive approach, so this property is
tested as a sequential property.

\paragraph{String Operation Behavior} String operation behavior properties are
verified by a set of tests that store the current test output for comparison
for future executions. There are 23 operations with associated tests for
each. The string operation classes are listed in
Table~\ref{tab:string-operations}.

\begin{table}[h]
  {\footnotesize
  \begin{tabular}{l l l l}
    \toprule
    \texttt{CharAt1} & \texttt{Postfix} & \texttt{Replace5} & \texttt{Split} \\
    \midrule
    \texttt{CharAt2} & \texttt{Prefix} & \texttt{Replace6} & \texttt{Substring} \\
    \midrule
    \texttt{Contains} & \texttt{Replace1} & \texttt{Reverse} & \texttt{ToLowerCase} \\
    \midrule
    \texttt{Delete} & \texttt{Replace2} & \texttt{SetCharAt1} & \texttt{ToUpperCase} \\
    \midrule
    \texttt{DeleteCharAt} & \texttt{Replace3} & \texttt{SetCharAt2} & \texttt{Trim} \\
    \midrule
    \texttt{Insert} & \texttt{Replace4} & \texttt{SetLength} & \\
    \bottomrule
  \end{tabular}
  }
  \caption{String Operations}%
  \label{tab:string-operations}
\end{table}

\paragraph{String Operation in Sequence} String operations should be able to be
used in sequence, i.e., operations can be chained together, with the automaton
returned being used as input to the next in sequence.  Assumptions about the
returned automata may not hold between operations in the sequence, worse, they
may not be detected by property checks.  Several tests were created to take
this possibility into account.  However, due to the large number of possible
combinations only a small subset of these tests were implemented.

\paragraph{Returned Automata Minimal} Several methods in the automata class
were commented as returning minimal automata.  These methods were tested in a
separate JUnit test case, as determining if an automata is minimal required
several steps.  These methods are listed in Table~\ref{tab:minimal-methods}.

\begin{table}[h]
  {\footnotesize
    \begin{tabular}{ll}
      \toprule
      \multicolumn{2}{c}{\texttt{MinimizationOperations.java}} \\
      \midrule
      \texttt{minimize()} & \texttt{minimizeHuffman()} \\
      \midrule
      \texttt{minimizeValmari()} & \texttt{minimizeBrzozowski()} \\
      \midrule
      \texttt{minimizeHopcroft()} & \\
      \bottomrule
    \end{tabular}
  }
  \caption{Minimal Methods}%
  \label{tab:minimal-methods}
\end{table}

\paragraph{Returned Automata Deterministic} Nine methods were commented as
returning deterministic automata.  Since this property could be tested by
accessing the \lstinline{isDeterministic()} method, all of them were
instrumented with assertions. These methods are listed in
Table~\ref{tab:deterministic-methods}.

\begin{table}[h]
  {\footnotesize
    \begin{tabular}{ll}
      \toprule
      \multicolumn{2}{c}{\texttt{BasicAutomata.java}} \\
      \midrule
      \texttt{makeEmtpyString()} & \texttt{makeCharSet()} \\
      \midrule
      \texttt{makeAnyString()} & \texttt{makeString()} \\
      \midrule
      \texttt{makeAnyChar()} & \texttt{makeStringUnion()} \\
      \midrule
      \texttt{makeChar()} & \texttt{makeStringMatcher()} \\
      \midrule
      \texttt{makeCharRange()} & \\
      \bottomrule
    \end{tabular}
  }
  \caption{Deterministic Methods}%
  \label{tab:deterministic-methods}
\end{table}

\paragraph{Automata Arguments Immutable} Automata have an instance variable
that indicates whether or not an operation is allowed to change it.  By
default, this variable is set to false, which should indicate to string
operations that the automata is to be considered immutable.  It was found that
most operations do not check this flag, but do in fact clone arguments in order
that they remain unchanged inside the method.  Considering this, a desirable
property for all string operations is to treat all automata arguments as
immutable.  Checks were placed in the JUnit test cases to capture the state of
arguments passed to operations in order to compare them after an operation
completed.

\paragraph{Automata Hashcodes Set} Each automaton contains a basic hashcode
that is a function of the number of states and transitions comprising the
automaton.  It is used as part of the equality comparison, and regardless if
the actual state and transitions in two automata are identical, a hashcode
inequality will cause the comparison to return false.  Since the hashcode
accessor method recalculates a cleared hashcode it was difficult to determine
if the code was correct.  Checks for a hashcode of \lstinline{0} were
implemented as assertions in the automata operations but sequential checks were
implemented to look for the \lstinline{Automaton.recalculateHashCode()} method
call after an occurrence of \lstinline{Automaton.clearHashCode()}.

\paragraph{Consistent Minimization Algorithm} The automaton class provides
three different minimization algorithms.  A flag is set in the automaton that
indicates which algorithm to use when \lstinline{Automaton.minimize()} is
called.  Automata with different minimization algorithms were considered to be
usable together as arguments to a single operation.  Since there is only one
minimal representation of a deterministic automata, each algorithm should
produce identical results.

\paragraph{Automata Set Operations Correct} The basic set operations of union,
intersection and minus are used repeatedly throughout the string operations.
These operations have well defined behaviors that can be verified.

\paragraph{Argument Automata Cloning (MOP)} Automata passed as arguments should
not be changed.  Immediate cloning inside the \lstinline{op()} method protects
the arguments.

\begin{table}[h]
  {\footnotesize
    \begin{tabular}{ll}
      \toprule
      \multirow{3}{*}{Events} & \texttt{UnaryOperation.op()} \\
                              & \texttt{BinaryOperation.op()} \\
                              & \texttt{Automaton.clone()} \\
      \midrule
      Regular Expression & \texttt{c* o c+} \\
      \bottomrule
    \end{tabular}
    \caption{Argument Automata Cloning}%
    \label{tab:arg-automata-clone}
  }
\end{table}

\paragraph{Determinize Before Minimize (MOP)} Automata should be deterministic
before they are minimized. Calling \lstinline{determinize()} before
\lstinline{minimize()} ensures this.

\begin{table}[h]
  {\footnotesize
    \begin{tabular}{ll}
      \toprule
      \multirow{2}{*}{Events} & \texttt{Automaton.determinize()} \\
                              & \texttt{Automaton.minimize()} \\
      \midrule
      Regular Expression & \texttt{(d | d m)*} \\
      \bottomrule
    \end{tabular}
    \caption{Determinize before Minimize}%
    \label{tab:determinize-before-minimize}
  }
\end{table}

\paragraph{Recompute Hashcode after Minimize (MOP)} Minimize calls do not have
a specific \lstinline{clearHashCode} call within them.  A call to
\lstinline{recomputeHashCode()} ensures it is correctly set.

\begin{table}[h]
  {\footnotesize
    \begin{tabular}{ll}
      \toprule
      \multirow{2}{*}{Events} & \texttt{Automaton.minimize()} \\
                              & \texttt{Automaton.recomputeHashCode} \\
      \midrule
      Regular Expression & \texttt{(r | m r)*} \\
      \bottomrule
    \end{tabular}
    \caption{Recompute Hashcode after Minimize}%
    \label{tab:recompute-hash-after-minimize}
  }
\end{table}

\paragraph{Recompute Hashcode After Clearing Hashcode (MOP)} The hashcode needs
to be recomputed after it is explicitly cleared.

\begin{table}[h]
  {\footnotesize
    \begin{tabular}{ll}
      \toprule
      \multirow{2}{*}{Events} & \texttt{Automaton.clearHashCode()} \\
                              & \texttt{Automaton.recomputeHashCode()} \\
      \midrule
      Regular Expression & \texttt{(r | c+ r+)*} \\
      \bottomrule
    \end{tabular}
    \caption{Recompute Hashcode After Clearing Hashcode}%
    \label{tab:recompute-hash-after-clear-hash}
  }
\end{table}

\paragraph{Restore Invariant After State Access (MOP)} Modifying states or
transitions requires a call to \lstinline{restoreInvariant}.  If the states are
accessed, it is possible they have been modified.

\begin{table}[h]
  {\footnotesize
    \begin{tabular}{ll}
      \toprule
      \multirow{2}{*}{Events} & \texttt{Automaton.restoreInvariant()} \\
                              & \texttt{Automaton.getStates()} \\
      \midrule
      Regular Expression & \texttt{(rI* | gS+ rI+)*} \\
      \midrule
    \end{tabular}
    \caption{Restore Invariant after State Access}%
    \label{tab:restore-invariant-after-state-access}
  }
\end{table}

\section{Testing}\label{sec:testing}

The software under test in the project is primarily for use in a program used
for analysis of string constraints in Java, the Java String Analyzer
(JSA)~\cite{strings2003}.  The string operations can also be used to implement
other string constraint analysis programs, and the automata package can be used
to implement automata outside of the string constraint analysis context.
Therefore, no single interface or program was available for us to test.
Testing was accomplished by creating JUnit tests that executed string
operations in isolation or as part of a sequence.  In order to more thoroughly
test the automaton package, JUnit tests were created that operated directly
with the automata instead of through the string operations.

\subsection{Test Cases}\label{sec:testing-cases}

Each string operation is designed to modify an automata which represents a
string or set of strings.  String operations extend either the
\lstinline{UnaryOperation} or the \lstinline{BinaryOperation} classes.  These
classes provide a single public method, \lstinline{op()}, that takes one or two
automata as arguments.  Arguments other than automata are passed to the
operation through the constructor.  String operations take a variety of other
argument types such as char and string in this manner.  The string operations
are used by instantiating an operation with arguments followed by calling the
\lstinline{op()} method with the target automaton as arguments.  The return
type of \lstinline{op()} is Automaton.

In order to fully test the operations, characteristics of an automata such as
length, alphabet and whether or not they accept the empty string or are empty
were determined.  These characteristics were used in place of input parameters.
Individual automata were created with characteristic combinations determined
using two tools, Test Specification Language (TSL)~\cite{Ostrand_1988} and
Combining Arrays through Simulated Annealing
(CASA)~\cite{cohen96_combin_desig_approac_to_autom_test_gener}.  These
characteristics are shown in Table~\ref{tab:automata-characteristics}

\begin{table}[ht]
  {\footnotesize
    \begin{tabular}{lll}
      \toprule
      \multirow{3}{*}{Length} & MIN & 4 characters long \\
                              & MED & 100 characters long \\
                              & MAX & 200 characters long \\
      \midrule
      \multirow{3}{*}{Alphabet} & MIN & 4 characters \\
                              & MED & 26 characters \\
                              & MAX & All characters \\
      \midrule
      \multirow{2}{*}{Empty String} & YES & Accepts empty string \\
                              & NO & Rejects empty string \\
      \midrule
      \multirow{2}{*}{Empty} & YES & Rejects all strings \\
                              & NO & Accepts some strings \\
      \bottomrule
    \end{tabular}
  }
  \caption{Automata Characteristics}%
  \label{tab:automata-characteristics}
\end{table}

TSL applies the Category Partition Method in order to determine test case
parameter values.  Parameters are first categorized and further partitioned
within each category.  These are then analyzed by the TSL tool and a
comprehensive list of parameter value combinations produced.  Using the
categories and partitions except the Empty category defined in
Table~\ref{tab:automata-characteristics} as input to TSL generated \(18\)
combinations.  The Empty category was implemented as a separate, empty
automaton since being empty precluded having any other property.

CASA applies Combinatorial Interaction Testing (CIT) as a way of reducing the
number of required parameter value combinations.  Research has shown that a
majority of errors are the result of pairs of input parameter values.  This
means that it is possible to find a majority of errors by testing all possible
pairs of input parameters.  Using CASA with the categories for length, alphabet
and empty string produced \(9\) combinations.  These were matched to the
corresponding TSL cases and \(9\) automata definitions were created.  Each
string operation was tested with automata that fit these 9 definitions.

A similar process was attempted in order to create the string operation
sequences for testing.  Due to the large number of string operations the number
of test cases required to achieve pair-wise testing in a sequence of \(3\)
operations was \(529\) test cases.  Two of these combinations were implemented,
but the length of time required to run them on the various automata types
proved prohibitive.

Input and output files from the TSL and CASA processes are provided in the TSL
and CASA directories in the repository.

\subsection{Testing Correctness}\label{sec:testing-correctness}

Testing the correctness of the string operations was treated as an opportunity
to create a set of regression tests rather than implementing specific tests for
each string operation that targeted a specific behavior of a Java string
operation.  In order to create these regression tests, correct behavior is
assumed, and tests are created that are intended to verify continued correct
behavior.

In order to create the oracle of correct results, the ability to save the
resulting automata from all string operation tests was built into the JUnit
tests.  The resulting automata are serialized and stored in files under a
specific directory.  The tests can be run in a mode that creates the test result
files or uses the result files to verify correct behavior.

\section{Results}\label{sec:results}

\subsection{Coverage}\label{sec:results-coverage}

All but three of the string operations achieved 100\% coverage in the
\lstinline{op()} method.  The three operations where 100\% coverage of the
\lstinline{op()} method was not achieved expose weaknesses in our choice of
inputs.  In the case of \lstinline{ToUpperCase}, there is a branch that is
taken only with extended character sets.  The use of these character sets was
not considered when determining the characteristics of input automata.  There
are two special cases in \lstinline{Contains} that were not covered with the
input combinations: An input automata that is empty, and an input automata that
is infinite.  Finally, the \lstinline{Replace6} operation has a special case in
which the target of the replace is an empty string.

The \lstinline{op()} method in \lstinline{Replace5}, \lstinline{CharAt2}, and
\lstinline{Split} consist of a single statement that executes other operations.
This led to low instruction and line coverage for these operations.

Summarized coverage results are presented in Table~\ref{tab:code-coverage}.  A
full coverage report in \texttt{HTML} format is provided in the \path{coverage}
directory in the repository.

\begin{table}[ht]
  {\scriptsize
    \begin{tabular}{lrrrr}
      \toprule
      \bf{Class} & \bf{Instruction} & \bf{Branch} & \bf{Line} & \bf{op()} \\
      \midrule
      Automaton & 74\% & 77\% & 74\% & NA \\
      \midrule
      MinimizationOperations & 99\% & 93\% & 97\% & NA \\
      \midrule
      CharAt1 & 92\% & 92\% & 90\% & 100\% \\
      \midrule
      CharAt2 & 50\% & NA & 40\% & 100\% \\
      \midrule
      Contains & 86\% & 79\% & 76\% & 88\% \\
      \midrule
      Delete & 89\% & 100\% & 83\% & 100\% \\
      \midrule
      DeleteCharAt & 89\% & 100\% & 82\% & 100\% \\
      \midrule
      Insert & 86\% & 100\% & 77\% & 100\% \\
      \midrule
      Postfix & 77\% & 100\% & 69\% & 100\% \\
      \midrule
      Prefix & 77\% & 100\% & 69\% & 100\% \\
      \midrule
      Replace1 & 65\% & 60\% & 69\% & 100\% \\
      \midrule
      Replace2 & 64\% & 57\% & 62\% & 100\% \\
      \midrule
      Replace3 & 62\% & 50\% & 62\% & 100\% \\
      \midrule
      Replace4 & 81\% & 100\% & 72\% & 100\% \\
      \midrule
      Replace5 & 32\% & NA & 38\% & 100\% \\
      \midrule
      Replace6 & 80\% & 72\% & 84\% & 98\% \\
      \midrule
      Reverse & 88\% & 100\% & 81\% & 100\% \\
      \midrule
      SetCharAt1 & 75\% & 75\% & 73\% & 100\% \\
      \midrule
      SetCharAt2 & 89\% & 100\% & 81\% & 100\% \\
      \midrule
      SetLength & 81\% & 100\% & 76\% & 100\% \\
      \midrule
      Split & 50\% & NA & 40\% & 100\% \\
      \midrule
      Substring & 81\% & 100\% & 74\% & 100\% \\
      \midrule
      ToLowerCase & 83\% & 90\% & 77\% & 100\% \\
      \midrule
      ToUpperCase & 64\% & 71\% & 63\% & 72\% \\
      \midrule
      Trim & 96\% & 87\% & 94\% & 100\% \\
      \bottomrule
    \end{tabular}
  }
  \caption{Code Coverage}%
  \label{tab:code-coverage}
\end{table}

\subsection{Static Analysis}\label{sec:results-static-analysis}

Coverity identified \(12\) defects in \(2\) categories and a defect density of
\(0.63\).  These were easily examined and classified without the need for
further tuning.  A general observation is that the default Java checks appear
to be very conservative in what they identify as problems, which could be an
attempt to avoid large amounts of false positives that require verification by
developers.  The most interesting of these defects were resource leaks
identified in the serialization and deserialization of several classes in the
automaton package.  These defects could have been the source of specific
problems that occurred when attempting to serialize and deserialize automata.
Repeated serialization and deserialization of large automata led to a stack
overflow exception.  Examination of the stack showed that the error happens at
a point when input and output streams are in use.

\subsection{Hash Code Not Set}\label{sec:results-hashcode-not-set}

Sequential checks put in place to observe the sequence of
\lstinline{clearHashcode()} and \lstinline{recomputeHashcode()} calls
identified a string operation that failed to correctly set the hash code before
returning.  The accessor method for the hashcode will recompute a hashcode of
zero before returning a result.  This behavior masks the fact that an operation
failed to set the hashcode correctly.  However, the instance variable that
holds the hashcode in the automaton class is marked package private, which
gives code inside the automaton package visibility to the value of this
variable.  It is possible, therefore, that code exists within the automaton
package that accesses this value directly and could possibly obtain an
incorrect hashcode.

\subsection{Assertion Failure}\label{sec:results-assertion-failure}

Certain automata failed a deterministic assertion check in the string operation
\lstinline{Replace6()}.  This assertion was present in the source code before
analysis began.  The assertion asserts that an automata passed as an argument is
deterministic after it was cloned.  It is difficult to say if the intent was to
make sure that an argument was deterministic or that the clone operation
returned a deterministic automata.  In either case, it appears that some
automata operations that are used to return new automata or modify existing
ones do not return deterministic automata.  In this case the functions related
to repeating a given automata a number of times in order to make one that
accepts longer strings does not return a deterministic automata when it appears
it should.

\subsection{Determinize and Minimize Calls}%
\label{sec:results-determinize-minimize}

The sequence of determinizing an automaton before minimizing was difficult to
observe through MOP\@.  A call to \lstinline{determinize()} is the first
statement within the \lstinline{minimize()} method.  While this would normally
ensure that an automata was deterministic before the minimization occurs, it
also led to multiple instances of a call sequence such as
\lstinline{determinize()}-\lstinline{minimize()}-\lstinline{determinize()}.
This is not necessarily a defect, but does indicate that the performance of the
libraries could be improved with careful consideration of how the
determinization and minimization occur.

\section{Conclusion}

The \lstinline{automaton} and \lstinline{string} libraries tend to be very
robust, as expected from a mature pair of libraries.  There are certain issues
of this age and the overall maintenance, but this did not impact the overall
examination to a large degree.

There are, however, several standout results of the two libraries.  There is a
bug in the handling of hash codes.  Furthermore, there is a defect in the
instance variable accessor modifiers that may allow more problematic code in
future versions.  Some of the assumptions that are encoded by the assertions do
not universally hold for all combinations of automata and operations.  Overall,
a deeper inspection of the method call sequence may yield further improvements
of both code organization and performance.

\bibliographystyle{plain}
\bibliography{references}

\end{document}

% LocalWords:  JML AspectJ JUnit accessor deserialization deserialize Coverity
% LocalWords:  dereferences determinized determinizing automata hashcode TSL
% LocalWords:  Determinize Postfix CASA Combinatorial Substring
