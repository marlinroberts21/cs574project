\documentclass[letterpaper,11pt]{article}
\usepackage{geometry}
\usepackage{listings}
\usepackage[utf8]{inputenc}
\usepackage[T1]{fontenc}
\usepackage{graphicx}
\usepackage{grffile}
\usepackage{longtable}
\usepackage{wrapfig}
\usepackage{rotating}
\usepackage[normalem]{ulem}
\usepackage{amsmath}
\usepackage{textcomp}
\usepackage{amssymb}
\usepackage{capt-of}
\usepackage{hyperref}
\title{CS574 Project Proposal}
\date{\today}
\hypersetup{
  pdfauthor={Kenny Ballou \& Marlin Roberts},
  pdftitle={Project Proposal},
  pdfkeywords={},
  pdfsubject={},
  pdflang={English}}
\author{Kenny Ballou \& Marlin Roberts}

\begin{document}

\maketitle{}
\tableofcontents{}

\section{Introduction}

\subsection{Application}

The application is the string constraint solving evaluation framework located
at: \url{https://github.com/BoiseState/string-constraint-counting}.

\subsubsection{Description}

This codebase provides a framework for demonstrating the applicability of
various automata types used to represent symbolic strings in a string
constraint solving system.  It currently has implementations for the following
automata types:

\begin{itemize}
\item{Bounded}
\item{Acyclic}
\item{Acyclic Weighted}
\end{itemize}

A modified solver based on Java String Analyzer (JSA)~\cite{strings2003} is
provided that can be used with any of the three automata types to solve various
benchmark constraints as well as a set of constraints obtained from analysis of
actual programs. It can output satisfiability results as well as model count
results for comparing the accuracy of the automata models.  While the current
codebase is largely compliant with the properties and functionality identified,
the instrumentation developed here could be used to ensure that automata and
automata factory classes developed in the future adhere to the requirements of
the framework.

\subsection{Metrics}

\begin{itemize}
\item[SLOC]{approximately 14,000}
\item[Source Files]{approximately 75}
\item[Classes]{}
\item[Methods]{}
\end{itemize}

\subsection{Properties}

\begin{enumerate}
\item{Operations involving two automata can only be performed on automata of
    the same type.}
\item{Automata should be minimized before operations occur.}
\item{Automata are required to be determinized before they are minimized.}
\item{Automata should be normalized before they are determinized.}
\item{Automata are required to have dead states removed before normalization.}
\item{Chosen solver type has to support the reporter requirements.}
\end{enumerate}

\subsection{Functionality}

\subsubsection{Desirable}

\begin{itemize}
\item{Repeated calls to minimize should return unchanged automata if it has not
    been modified.}
\item{Operations that are not implemented should return unmodified automata.}
\item{Automata are required to provide a clone method, with appropriate
    functionality.}
\end{itemize}
\subsubsection{Undesirable}

\nocite{*}
\bibliographystyle{plain}
\bibliography{references}

\end{document}
